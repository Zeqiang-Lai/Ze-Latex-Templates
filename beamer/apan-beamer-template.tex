%!TEX TS-program = xelatex
%!TEX encoding = UTF-8 Unicode

\documentclass[10pt, aspectratio=1610]{beamer}

%% 西文字配置
\linespread{1.2}
\usepackage{arevmath}
\usepackage[no-math]{fontspec}
\setmainfont[Mapping=tex-text,LetterSpace=-1.5]{DejaVu Sans}
\setsansfont[Mapping=tex-text,LetterSpace=-1.5]{DejaVu Sans}
\setmonofont{DejaVu Sans Mono}
    % 欲採用 Inconsolata 為西文等寬字請採用下三列
    % \setmonofont[Scale=MatchLowercase,
    %   BoldFont={Inconsolata},BoldFeatures={FakeBold=5},
    %   ItalicFont={Inconsolata},ItalicFeatures={FakeSlant=0.2}]{Inconsolata}
\usefonttheme{professionalfonts}

%% 中文字配置
\usepackage[
    CJKmath=true, indentfirst=false, PunctStyle={quanjiao},
    CheckSingle=true, SlantFont, BoldFont]
    {xeCJK} 

% 配方 1: cwTeX Q fonts
    \setCJKmainfont[Scale=1.2,BoldFeatures={FakeBold=2}]
        {cwTeX Q Hei Bold}
    \setCJKmonofont[Scale=1.2,BoldFeatures={FakeBold=4},FakeStretch=1.015319]
        {cwTeX Q Yuan} 
    % 欲採用 Inconsolata 為西文等寬字請採用下二列
    % \setCJKmonofont[Scale=1.2,BoldFeatures={FakeBold=4},
    %   FakeStretch=0.9869091]{cwTeX Q Yuan} 

% 配方 2: Noto Sans CJK JP
    %\setCJKmainfont[Scale=1.204102, BoldFeatures={}]
    %    {Noto Sans CJK JP}
	% \setCJKmonofont[Scale=1.204102, BoldFeatures={}]
	%     {Noto Sans Mono CJK JP} 
    % 欲採用 Inconsolata 為西文等寬字請採用下二列
    %\setCJKmonofont[Scale=1.17811, BoldFeatures={}]
    %    {Noto Sans Mono CJK JP} 


%% 仿西式註腳風格
\let\oldfootnote\footnote
\renewcommand\footnote[1]{\hspace{-0.7em}\oldfootnote{\ignorespaces#1}\hspace{0.5em}}

%% 非必須 package
\usepackage{fancyvrb}
\usepackage{pstricks}

\begin{document}

\title{阿盤的Beamer中英字型配方}
\author{Chen-Pan Liao}
\institute{\texttt{<andrew.43@gmail.com>}}

\begin{frame}
\titlepage
\end{frame}

\section{授權條款}

\begin{frame}{授權條款}
\begin{center}

% start of logo of cc-by-sa
%LaTeX with PSTricks extensions
%%Creator: inkscape 0.48.2
%%Please note this file requires PSTricks extensions
\psset{xunit=.15pt,yunit=.15pt,runit=.15pt}
\begin{pspicture}(760.21386719,234.5014801)
	{
	\newrgbcolor{curcolor}{0 0 0}
	\pscustom[linestyle=none,fillstyle=solid,fillcolor=curcolor]
	{
	\newpath
	\moveto(379.897101,234.5014621)
	\curveto(412.829941,234.5014621)(440.611021,223.1977721)(463.225841,200.5830421)
	\curveto(485.972491,177.8290751)(497.356701,150.0516351)(497.356701,117.2507451)
	\curveto(497.356701,84.3069651)(486.185001,56.8812851)(463.848661,34.9700451)
	\curveto(440.120061,11.6554751)(412.137511,0.0000251)(379.897101,0.0000251)
	\curveto(348.206461,0.0000251)(320.711261,11.5199051)(297.404021,34.5523451)
	\curveto(274.375261,57.5847751)(262.855331,85.1460351)(262.855331,117.2507451)
	\curveto(262.855331,149.3517851)(274.375261,177.1292451)(297.404021,200.5793621)
	\curveto(320.018671,223.1977721)(347.510341,234.5014621)(379.897101,234.5014621)
	\closepath
	\moveto(380.314891,213.3523721)
	\curveto(353.655041,213.3523721)(331.113551,203.9979721)(312.686851,185.2964851)
	\curveto(293.560381,165.7522451)(284.000701,143.0678751)(284.000701,117.2470851)
	\curveto(284.000701,91.2833851)(293.490741,68.8078951)(312.470631,49.8352451)
	\curveto(331.454351,30.8479551)(354.065351,21.3616451)(380.307561,21.3616451)
	\curveto(406.406801,21.3616451)(429.164401,30.9138851)(448.565771,50.0440951)
	\curveto(466.992511,67.7709351)(476.203931,90.1731551)(476.203931,117.2543851)
	\curveto(476.203931,143.9142951)(466.849471,166.5913251)(448.151831,185.3001551)
	\curveto(429.450231,204.0016321)(406.835571,213.3523721)(380.314891,213.3523721)
	\closepath
	\moveto(411.723391,146.3545751)
	\lineto(411.723391,98.4100051)
	\lineto(398.327471,98.4100051)
	\lineto(398.327471,41.4628151)
	\lineto(361.891771,41.4628151)
	\lineto(361.891771,98.4063551)
	\lineto(348.496021,98.4063551)
	\lineto(348.496021,146.3545751)
	\curveto(348.496021,148.4504351)(349.228791,150.2275051)(350.690811,151.6931451)
	\curveto(352.160081,153.1551051)(353.940771,153.8915951)(356.029241,153.8915951)
	\lineto(404.189991,153.8915951)
	\curveto(406.142971,153.8915951)(407.890741,153.1587751)(409.422341,151.6931451)
	\curveto(410.950291,150.2275051)(411.723391,148.4467651)(411.723391,146.3545751)
	\closepath
	\moveto(363.771521,176.5026751)
	\curveto(363.771521,187.5242551)(369.212661,193.0423521)(380.106071,193.0423521)
	\curveto(390.999391,193.0423521)(396.436811,187.5315651)(396.436811,176.5026751)
	\curveto(396.436811,165.6166851)(390.991991,160.1718551)(380.106071,160.1718551)
	\curveto(369.220011,160.1718551)(363.771521,165.6166851)(363.771521,176.5026751)
	\closepath
	}
	}
	{
	\newrgbcolor{curcolor}{0 0 0}
	\pscustom[linestyle=none,fillstyle=solid,fillcolor=curcolor]
	{
	\newpath
	\moveto(642.754261,234.5014801)
	\curveto(675.551371,234.5014801)(703.325271,223.1245001)(726.082831,200.3705401)
	\curveto(748.829461,177.7557851)(760.213881,150.0516551)(760.213881,117.2507751)
	\curveto(760.213881,84.4425551)(749.041981,56.9509251)(726.705701,34.7575751)
	\curveto(703.120091,11.5859251)(675.130031,0.0000051)(642.754261,0.0000051)
	\curveto(611.063671,0.0000051)(583.568341,11.5162451)(560.261081,34.5487051)
	\curveto(537.232321,57.5774651)(525.712461,85.1387251)(525.712461,117.2470951)
	\curveto(525.712461,149.2088951)(537.232321,176.9167151)(560.261081,200.3668601)
	\curveto(583.011331,223.1245001)(610.510321,234.5014801)(642.754261,234.5014801)
	\closepath
	\moveto(643.171911,213.3523901)
	\curveto(616.512051,213.3523901)(593.970671,203.9283501)(575.543921,185.0839651)
	\curveto(556.417391,165.6789851)(546.857871,143.0679051)(546.857871,117.2507751)
	\curveto(546.857871,91.1441651)(556.347871,68.6759651)(575.327831,49.8315851)
	\curveto(594.311351,30.8442951)(616.922561,21.3580251)(643.164631,21.3580251)
	\curveto(669.264011,21.3580251)(692.021531,30.9175851)(711.422741,50.0404651)
	\curveto(729.849551,67.9065351)(739.061051,90.3087751)(739.061051,117.2507751)
	\curveto(739.061051,144.0462351)(729.706631,166.6609451)(711.008801,185.0839651)
	\curveto(692.442831,203.9356901)(669.828121,213.3523901)(643.171901,213.3523901)
	\closepath
	\moveto(590.826891,133.7941051)
	\curveto(593.058321,148.1719551)(598.847551,159.3034751)(608.202021,167.1885751)
	\curveto(617.552711,175.0736951)(628.925961,179.0162451)(642.329241,179.0162451)
	\curveto(660.748601,179.0162451)(675.412241,173.0804351)(686.298171,161.2234551)
	\curveto(697.184251,149.3591551)(702.629071,134.1421951)(702.629071,115.5799351)
	\curveto(702.629071,97.5709551)(696.975431,82.6031651)(685.675351,70.6729251)
	\curveto(674.360601,58.7426451)(659.711691,52.7702051)(641.698941,52.7702051)
	\curveto(628.438711,52.7702051)(616.992091,56.7457251)(607.362841,64.7041051)
	\curveto(597.729991,72.6625251)(591.940701,83.9662251)(589.984201,98.6225551)
	\lineto(619.512921,98.6225551)
	\curveto(620.209251,84.3839251)(628.794081,77.2646151)(645.267891,77.2646151)
	\curveto(653.497361,77.2646151)(660.133101,80.8260951)(665.156511,87.9417551)
	\curveto(670.187281,95.0573951)(672.700831,104.5547151)(672.700831,116.4153451)
	\curveto(672.700831,128.8402551)(670.396151,138.2936151)(665.794011,144.7863751)
	\curveto(661.184651,151.2754451)(654.559961,154.5218251)(645.898051,154.5218251)
	\curveto(630.259781,154.5218251)(621.469631,147.6150551)(619.516641,133.7977851)
	\lineto(628.101651,133.7977851)
	\lineto(604.864021,110.5564851)
	\lineto(581.622651,133.7977851)
	\lineto(590.826891,133.7957851)
	\lineto(590.826891,133.7957851)
	\closepath
	}
	}
	{
	\newrgbcolor{curcolor}{0 0 0}
	\pscustom[linestyle=none,fillstyle=solid,fillcolor=curcolor]
	{
	\newpath
	\moveto(117.03454,234.4996441)
	\curveto(149.83174,234.4996441)(177.75575,223.0493641)(200.784531,200.1598241)
	\curveto(211.806091,189.1346151)(220.185901,176.5338251)(225.909111,162.3684551)
	\curveto(231.628841,148.1994351)(234.497741,133.1620451)(234.497741,117.2489251)
	\curveto(234.497741,101.1965451)(231.665361,86.1554951)(226.019051,72.1330251)
	\curveto(220.365351,58.1069251)(212.022311,45.7149751)(201.000741,34.9682151)
	\curveto(189.557821,23.6645151)(176.57589,15.0062851)(162.05518,9.0045051)
	\curveto(147.54174,3.0027251)(132.53365,0.0018451)(117.04183,0.0018451)
	\curveto(101.5501,0.0018451)(86.7178903,2.9624251)(72.54883,8.9019051)
	\curveto(58.38347,14.8304051)(45.68011,23.4153551)(34.44244,34.6494251)
	\curveto(23.20466,45.8835151)(14.65635,58.5539351)(8.79381,72.6533451)
	\curveto(2.93127,86.7527251)(0,101.6142751)(0,117.2489251)
	\curveto(0,132.7443451)(2.96425,147.6425051)(8.89638,161.9507651)
	\curveto(14.82854,176.2590051)(23.45015,189.0649851)(34.75386,200.3723641)
	\curveto(57.08638,223.1189841)(84.5121503,234.4996441)(117.03454,234.4996441)
	\closepath
	\moveto(117.45961,213.3505541)
	\curveto(90.6604003,213.3505541)(68.11534,203.9961341)(49.8279,185.2946551)
	\curveto(40.61271,175.9402351)(33.53009,165.4389751)(28.57622,153.7835151)
	\curveto(23.61498,142.1280751)(21.14174,129.9486251)(21.14174,117.2452451)
	\curveto(21.14174,104.6810851)(23.61498,92.5712951)(28.57622,80.9231651)
	\curveto(33.53366,69.2603951)(40.61271,58.8617151)(49.8279,49.7198251)
	\curveto(59.03937,40.5742651)(69.43443,33.6014851)(81.0275203,28.7795751)
	\curveto(92.6096803,23.9649751)(104.75615,21.5576751)(117.45961,21.5576751)
	\curveto(130.02007,21.5576751)(142.19209,23.9942951)(154.00154,28.8821551)
	\curveto(165.79611,33.7737351)(176.42939,40.8160951)(185.926711,50.0275951)
	\curveto(204.210461,67.8936651)(213.348641,90.2958951)(213.348641,117.2415851)
	\curveto(213.348641,130.2234251)(210.974301,142.5054751)(206.229391,154.0912951)
	\curveto(201.491721,165.6771251)(194.577601,176.0025251)(185.512601,185.0821451)
	\curveto(166.65717,203.9265141)(143.98016,213.3505541)(117.45961,213.3505541)
	\closepath
	\moveto(115.99024,136.7162051)
	\lineto(100.28232,128.5489451)
	\curveto(98.6042,132.0335051)(96.54864,134.4811251)(94.10835,135.8771451)
	\curveto(91.6643803,137.2694851)(89.3962803,137.9693251)(87.3004903,137.9693251)
	\curveto(76.8358403,137.9693251)(71.59625,131.0625151)(71.59625,117.2415851)
	\curveto(71.59625,110.9613351)(72.9226,105.9415651)(75.5717303,102.1711951)
	\curveto(78.2245503,98.4008651)(82.13410029,96.5138451)(87.3004903,96.5138451)
	\curveto(94.14132,96.5138451)(98.95596,99.8664951)(101.75169,106.5644351)
	\lineto(116.19551,99.2362651)
	\curveto(113.12502,93.5092851)(108.86728,89.0098151)(103.42242,85.7304751)
	\curveto(97.98494,82.4474251)(91.9794803,80.8095851)(85.4171403,80.8095851)
	\curveto(74.9488903,80.8095851)(66.49947,84.0156551)(60.08001,90.4424651)
	\curveto(53.66058,96.8619351)(50.45073,105.7949651)(50.45073,117.2379051)
	\curveto(50.45073,128.4060351)(53.69723,137.2658151)(60.18629,143.8281951)
	\curveto(66.67534,150.3869151)(74.8755903,153.6699151)(84.7905903,153.6699151)
	\curveto(99.3114,153.6769151)(109.70632,148.0235751)(115.99024,136.7162051)
	\closepath
	\moveto(183.618271,136.7162051)
	\lineto(168.12287,128.5489451)
	\curveto(166.44835,132.0335051)(164.38551,134.4811251)(161.94522,135.8771451)
	\curveto(159.49758,137.2694851)(157.15629,137.9693251)(154.93211,137.9693251)
	\curveto(144.46389,137.9693251)(139.22417,131.0625151)(139.22417,117.2415851)
	\curveto(139.22417,110.9613351)(140.55432,105.9415651)(143.20344,102.1711951)
	\curveto(145.85258,98.4008651)(149.75845,96.5138451)(154.93211,96.5138451)
	\curveto(161.76567,96.5138451)(166.58401,99.8664951)(169.37235,106.5644351)
	\lineto(184.028721,99.2362651)
	\curveto(180.82254,93.5092851)(176.49162,89.0098151)(171.05415,85.7304751)
	\curveto(165.60929,82.4474251)(159.67715,80.8095851)(153.25769,80.8095851)
	\curveto(142.64644,80.8095851)(134.17142,84.0156551)(127.82526,90.4424651)
	\curveto(121.46437,96.8619351)(118.29129,105.7949651)(118.29129,117.2379051)
	\curveto(118.29129,128.4060351)(121.53399,137.2658151)(128.03042,143.8281951)
	\curveto(134.51592,150.3869151)(142.71615,153.6699151)(152.62748,153.6699151)
	\curveto(167.1446,153.6769151)(177.48095,148.0235751)(183.618271,136.7162051)
	\closepath
	}
	}
\end{pspicture}
% end of logo of cc-by-sa

\end{center}
\begin{itemize}
  \item 《\href{%
  http://www.scribd.com/doc/83967053/Apan-s-Beamer-Template%
}{阿盤的Beamer中英字型配方}》由 
  \href{http://apansharing.blogspot.tw/}{\alert{Chen-Pan Liao}}
  製作,%
  \footnote{\url{http://apansharing.blogspot.tw/}}
  以\href{http://creativecommons.org/licenses/by-sa/3.0/}%
  {\alert{創用CC姓名標示-- 相同方式分享3.0 Unported 授權條款}}釋出。%
  \footnote{\url{http://creativecommons.org/licenses/by-sa/3.0/}}\\

  \item 本文件不定期更新,請在\\
\url{http://www.slideshare.net/chenpanliao/apan-beamertemplate}\\
下載最新版本。

  \item 本文件之原始碼可在\\
        \url{https://github.com/chenpanliao/apan-beamer-template} 取得。
\end{itemize}
\end{frame}

\section{字型配方}

\subsection{原則}

\begin{frame}{字型配方原則}
\begin{itemize}
\item 以xelatex(Ver.~3.1415926-2.5-0.9999.3;TeX Live 2013)加上
      xeCJK package(Ver.~3.2.9)下運作。
\item 儘可能使用sans serif字體(包括數學式)且排除太纖細的字體,以利投影片閱讀。
\item 一律使用自由的字體。
\item 在等寬字體時儘可能維持中文字與英文字之寬度比例為$2:1$。
\end{itemize}
\end{frame}


\subsection{西文字型配置}

\begin{frame}{西文字型配置}
\begin{block}{原則}
\begin{itemize}
\item 採用DejaVu Sans與DejaVu Sans Mono系列(Ver.~2.34)。
\item 利用fontspec package(Ver.~2.3)引入並採用 \texttt{no-math} 選項。
\item 調整 \texttt{LetterSpacing} 以減少字母間距。
\end{itemize}
\end{block}
\begin{block}{結果}
\begin{description}
\item[一般字體] Normal text; \textbf{Bold text}; \textit{Italic text}
\item[等寬字體] {\ttfamily Normal text; \textbf{Bold text}; \textit{Italic text}}
\end{description}
\end{block}
\end{frame}

\begin{frame}{數學字型配置}
\begin{block}{原則}
\begin{itemize}
\item 使用arevmath package(Ver.~2006-05-31)為數字式字型以配合系出同源的DejaVu Sans。
\item 在beamer的設定中採用 \texttt{professionalfonts}。
\end{itemize}
\end{block}
\begin{block}{結果}
\[
\frac{1}{2\pi i}\int_\gamma f = \sum_{k=1}^m n(\gamma;a_k) \text{Res}(f;a_k).
\]
\[
\max\{|f(z)|:z\in G^-\}=\max \{|f(z)|:z\in \partial G \}.
\]
\boldmath\[  \alpha + b = 27 \]
\end{block}
\end{frame}

\subsection{中文字型配置}

\begin{frame}{中文字型配置}
  \newcommand{\aaatext}{%
    {\CJKsetecglue{}\ttfamily 這是\textbf{測試}\textit{文字}%
    \ttfamily 這是\textbf{測試}\textit{文字}}}
  \newcommand{\bbbtext}{%
  {\ttfamily ii11\textbf{EESS}\textit{mmnn}%
    \ttfamily ii11\textbf{EESS}\textit{mmnn}}}
  \newlength{\aaa}
  \newlength{\bbb}
  \settowidth{\aaa}{\aaatext}
  \settowidth{\bbb}{\bbbtext}
\begin{block}{原則}
\begin{itemize}
\item 內文採用cwTeX-Q-Hei-Bold(Ver.~0.4+)為主要字體及cwTeX-Q-Yuan(Ver.~0.4+)為等寬字體。
\item 利用xeCJK package引入。
\item 調整 \texttt{Scale} 以與DejaVu Sans配合。
\item 調整等寬字體的 \texttt{FakeStretch},
使與西文等寬字體成$2:1$之寬度比例。
\item 使用 \texttt{BoldFont} 與 \texttt{SlantFont} 以產生粗體與斜體。
\end{itemize}
\end{block}
\begin{block}{結果}
\begin{description}
\item[一般字體] 正常字;\textbf{粗體字};\textit{斜體字}。
\item[等寬字體] {\ttfamily 正常字;\textbf{粗體字};\textit{斜體字}。}\\
「\aaatext{}」(占用\the\aaa) \\
「\bbbtext{}」(占用\the\bbb)
\end{description}
\end{block}
\end{frame}


\subsection{相關資源}

\begin{frame}{相關資源}
\begin{block}{字體下載}
\begin{description}
\item[DejaVu] {\url{http://dejavu-fonts.org/wiki/Main_Page}}
\item[cwTeX-Q] {\url{https://github.com/l10n-tw/cwtex-q-fonts}}
\end{description}
\end{block}
\begin{block}{相關package}
\begin{description}
\item[fontspec] {\url{http://www.ctan.org/tex-archive/macros/xetex/latex/fontspec/}}
\item[xecjk] {\url{http://www.ctan.org/tex-archive/macros/xetex/latex/xecjk}}
\item[arevmath] {\url{http://www.ctan.org/tex-archive/fonts/arev/}}
\end{description}
\end{block}
\begin{block}{回應或討論}
歡迎到 \alert{\url{http://hyperrate.com/thread.php?tid=23496}} 回應、討論與指教。
\end{block}
\end{frame}

\section{原始碼}

\subsection{重點原始碼}

\begin{frame}[allowframebreaks, fragile]
\frametitle{本文件重點原始碼}
{
  % 在 Verbatim 中關閉 xeCJK 的中英自動間隔並啟用原始標點符號設定
  \CJKsetecglue{}\punctstyle{plain} 
  \fvset{fontsize=\footnotesize}
  \VerbatimInput[firstline=1,lastline=38]{apan-beamer-template.tex}
}
\end{frame}

\subsection{完整原始碼}
\begin{frame}[allowframebreaks]
\frametitle{本文件完整原始碼}
{
  % 在 Verbatim 中關閉 xeCJK 的中英自動間隔並啟用原始標點符號設定
  \CJKsetecglue{}\punctstyle{plain} 
  \fvset{fontsize=\scriptsize, baselinestretch=1.1}
  \VerbatimInput{apan-beamer-template.tex}
}
\end{frame}

\end{document}
