\documentclass[10pt]{article}
\usepackage{minted, ctex, url}
\usepackage{amsthm, amsmath, amssymb, amsfonts}
\usepackage{graphicx, float, xcolor, subcaption}
%\usepackage[margin=1in]{geometry}
\usepackage{hyperref}
\usepackage[a4paper,left=3cm,right=3cm,top=2.5cm,bottom=2.5cm]{geometry}
\usepackage[toc,page]{appendix}
\usepackage{rotating}
\usepackage[hyperref=true,backend=biber,sorting=none,backref=true]{biblatex}
\addbibresource{ref.bib}

\title{周报(2020-10-12)}
\author{赖泽强}
\newtheorem{theorem}{\hspace{2em}Theorem}[section]
\newtheorem{definition}{\hspace{2em}Definition}[section]
\newtheorem{lemma}{\hspace{2em}Lemma}[section]
\newtheorem{proposition}{\hspace{2em}Proposition}[section]
\newtheorem{corollary}{\hspace{2em}Corollary}[section]
\newcommand{\todo}[1]{{\color{red}[TODO: #1]}}

\newcommand{\F}{\mathbb F}
\newcommand{\E}{\mathbb E}
\newcommand{\R}{\mathbb R}
\newcommand{\Q}{\mathbb Q}
\newcommand{\Z}{\mathbb Z}
\newcommand{\N}{\mathbb N}
\newcommand{\Prob}{\mathbb P}

\newcommand{\textand}{\quad \text{and} \quad}
\newcommand{\textor}{\quad \text{or} \quad}

\DeclareMathOperator*{\argmax}{arg\,max}
\DeclareMathOperator*{\argmin}{arg\,min}
\newcommand{\norm}[1]{\left \Vert #1 \right \Vert}
\newcommand{\set}[1]{\left \{ #1 \right \}}
\newcommand{\tr}{\text{tr}} % trace
\newcommand{\Var}{\text{Var}} 
\newcommand{\diag}{\text{diag}} 
\newcommand{\inner}[2]{\left \langle #1, #2 \right \rangle}




\begin{document}
\maketitle
\tableofcontents


\section{本周工作}

\begin{enumerate}
\item
  阅读论文 ``Bring Old Photos Back to Life''。
\item
  学习VAE。
\item
  复习生成式模型相关知识,如GDA,朴素贝叶斯等。
\item
  复习概率论,概率模型相关知识。
\end{enumerate}

\section{下周计划}

\begin{enumerate}
\item
  阅读并尝试运行``Bring Old Photos Back to Life''的代码。
\item
  继续学习``Bring Old Photos Back to
  Life''论文里一些不懂的概念和数学知识。
\item
  阅读与``Bring Old Photos Back to Life''相关的论文。
\end{enumerate}

\section{论文总结}

主要只读了一篇论文。

\subsection{Bring Old Photos Back to Life}

这个工作\cite{wan2020bringing}\cite{wan2020old}的目标任务是修复老照片,即恢复老照片原来的样子。

具体来说,包括修复两大方面损伤:

\begin{enumerate}
\item
  非结构化损伤(Unstructured defects):包括胶片噪声,模糊,褪色等。
\item
  结构化损伤(Structured defects):包括折痕和污渍等。
\end{enumerate}

\subsubsection{相关工作}

之前也有一些工作尝试对老照片进行修复,但这些工作存在两个主要问题:

\begin{enumerate}
\item
  首先是Mixed degradation
  issue。老照片的损伤通常是多方面的,但是之前的工作要么是只针对某一方面进行修复,要么是使用一种方法对付所有方面。
\item
  其次,目前大部分的模型,包括深度学习模型,都是使用合成的数据对进行训练,这种模型对于真实数据的效果会大打折扣。CycleGAN虽然使用了真实数据,但是它使用的不匹配的数据进行训练。目前没有工作同时使用真实数据和配对数据(存疑,需要调研下)。
\end{enumerate}

\subsubsection{方法}

这篇论文同时使用真实老照片,合成老照片及其对应的ground
truth作为训练数据。

如图\ref{arch}所示,文中所述模型主要分为三部分:

\begin{itemize}
\item
  首先是两个VAE\cite{kingma2014autoencoding}负责将老照片和好照片分别映射到两个隐空间。

  \begin{itemize}
  \item
    合成老照片和真实老照片公用一个VAE。
  \item
    VAE以非监督的形式单独训练,训练好后固定不变。
  \end{itemize}
\item
  然后是一个翻译网络T,负责将老照片的隐空间映射到好照片的隐空间。

  \begin{itemize}
  \item
    为了处理Structured
    defects(通常需要更多的上下文),作者引入了Partial nonlocal block\cite{wang2018nonlocal}。
  \end{itemize}
\end{itemize}

\begin{figure}
\centering
\includegraphics[scale=0.6]{/Users/laizeqiang/Documents/Study/BIT-Graduate/周报/2020-10-12/imgs/arch.png}
\caption{网络结构}
\label{arch}
\end{figure}

\subsubsection{重要公式解读}

\paragraph{VAE}~{}
\newline

论文所用的VAE Loss前两项和传统的一致,最后一项叫 least-square loss
(LSGAN)\cite{mao2017squares},用于解决VAE中的over-smooth issue。

\begin{equation}
\begin{aligned}
\mathcal{L}_{\mathrm{VAE}_{1}}(r) &=\mathrm{KL}\left(E_{\mathcal{R}, \mathcal{X}}\left(z_{r} \mid r\right) \| \mathcal{N}(0, I)\right) \\
&+\alpha \mathbb{E}_{z_{r} \sim E_{\mathcal{R}, \mathcal{X}}\left(z_{r} \mid r\right)}\left[\left\|G_{\mathcal{R}, \mathcal{X}}\left(r_{\mathcal{R} \rightarrow \mathcal{R}} \mid z_{r}\right)-r\right\|_{1}\right] \\
&+\mathcal{L}_{\mathrm{VAE}_{1}, \mathrm{GAN}}(r)
\end{aligned}
\end{equation}

为了更好缩小真实老照片和合成老照片的domain
gap,作者还引入了一个Adversarial
loss训练一个Discriminator,\(D_{r,x}\)越大,则Discriminator认为这个feature越接近真实老照片的feature。

\begin{equation}
\begin{aligned}
\mathcal{L}_{\mathrm{VAE}_{1}, \mathrm{GAN}}^{\text {latent }}(r, x)=& \mathbb{E}_{x \sim \mathcal{X}}\left[D_{\mathcal{R}, \mathcal{X}}\left(E_{\mathcal{R}, \mathcal{X}}(x)\right)^{2}\right] \\
&+\mathbb{E}_{r \sim \mathcal{R}}\left[\left(1-D_{\mathcal{R}, \mathcal{X}}\left(E_{\mathcal{R}, \mathcal{X}}(r)\right)\right)^{2}\right]
\end{aligned}
\end{equation}

下面的就是第一个VAE 总的Loss:

\[\min _{E_{\mathcal{R}, \mathcal{X}}, G_{\mathcal{R}, \mathcal{X}}} \max _{D_{\mathcal{R}, \mathcal{X}}} \mathcal{L}_{\mathrm{VAE}_{1}}(r)+\mathcal{L}_{\mathrm{VAE}_{1}}(x)+\mathcal{L}_{\mathrm{VAE}_{1}, \mathrm{GAN}}^{\text {latent }}(r, x)\]

\paragraph{Mapping}~{}
\newline

\[\mathcal{L}_{\mathcal{T}}(x, y)=\lambda_{1} \mathcal{L}_{\mathcal{T}, \ell_{1}}+\mathcal{L}_{\mathcal{T}, \mathrm{GAN}}+\lambda_{2} \mathcal{L}_{\mathrm{FM}}\]

\[\left.\mathcal{L}_{\mathcal{T}, \ell_{1}}=\mathbb{E} \| \mathcal{T}\left(z_{x}\right)-z_{y}\right) \|_{1}\]

这个Loss用于训练翻译网络T,第一部分是隐向量的L1 Loss,第二部分也是
least-square loss (LSGAN),第三部分是feature matching loss 用于
stabilize the GAN training. \textbf{后两部分目前没看懂}。

\paragraph{Multiple degradation restoration}~{}
\newline

在基础的Mapping网络中,Resnet Block使用的CNN
filter只能关注到local的feature,但是一些structured
defect,比如褶皱等,必须要有更大范围的上下文信息才能够进行修复(你不可能靠周围的褶皱修复褶皱)。

因此,作者引入了一个nonlocal block,之所以叫partial是因为nonlocal
block的原文里使用的是所有像素的信息,这里作者只关注非褶皱像素(作者还引入了一个defect
region detection network标出一个defect mask,即下面的m,m=1表示是defect
region)。

下面这个公式是nonlocal block原文里给出的公式,简单来说nonlocal
block就是将某个像素的特征值就是其他像素特征值的加权平均。公式里的\texttt{f}就是计算权值的一个函数,g则是计算特征值的函数,C(x)是一个正则项。

\[\mathbf{y}_{i}=\frac{1}{\mathcal{C}(\mathbf{x})} \sum_{\forall j} f\left(\mathbf{x}_{i}, \mathbf{x}_{j}\right) g\left(\mathbf{x}_{j}\right)\]

下面的公式都是本篇论文里的公式:

\begin{itemize}
\item
  \(s_{i,j}\)相当于计算权值,从公式里不难看出,这个权值对于defect
  region里的像素是0,即我们不关注defect region里的像素的特征值。
\item
  \(f_{i,j}\)用于计算两个像素之间的像素度,显然,越相似的像素我们应该越关注。
\item
  第三个公式几乎就是nonlocal block原文中的本文形式;
\item
  最后一个公式表示,对于defect region里的像素,我们采用nonlocal
  block计算出来的特征值,对于其他像素,特征值保持不变。
\end{itemize}

\[s_{i, j}=\left(1-m_{j}\right) f_{i, j} / \sum_{\forall k}\left(1-m_{k}\right) f_{i, k}\]

\[f_{i, j}=\exp \left(\theta\left(F_{i}\right)^{T} \cdot \phi\left(F_{j}\right)\right)\]

\[O_{i}=\nu\left(\sum_{\forall j} s_{i, j} \mu\left(F_{j}\right)\right)\]

\[F_{\text {fuse}}=(1-m) \odot \rho_{\text {local }}(F)+m \odot \rho_{\text {global }}(O)\]

\subsubsection{主要贡献}

\begin{itemize}
\item
  提出了一种比较有效的,在domain
  translation问题中,使用未配对数据的方法。
\item
  这个方法有机会使用到别的domain
  translation问题上,比如降噪,可以结合魏学长之前的工作进行扩展。
\end{itemize}

\printbibliography

\end{document}
